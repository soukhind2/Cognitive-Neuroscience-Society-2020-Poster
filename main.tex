%% %%%%%%%%%%%%%%%%%%%%%%%%%%%%%%%%%%%%%%%%%%%%%%%%%
%% Template for a conference paper, prepared for the
%% Food and Resource Economics Department - IFAS
%% UNIVERSITY OF FLORIDA
%% %%%%%%%%%%%%%%%%%%%%%%%%%%%%%%%%%%%%%%%%%%%%%%%%%
%% Version 1.0 // November 2019
%% %%%%%%%%%%%%%%%%%%%%%%%%%%%%%%%%%%%%%%%%%%%%%%%%%
%% Ariel Soto-Caro
%%  - asotocaro@ufl.edu
%%  - arielsotocaro@gmail.com
%% %%%%%%%%%%%%%%%%%%%%%%%%%%%%%%%%%%%%%%%%%%%%%%%%%
\documentclass[11pt]{article}
\usepackage{UF_FRED_paper_style}
\usepackage{subcaption} 

\usepackage{lipsum}  %% Package to create dummy text (comment or erase before start)

%% ===============================================
%% Setting the line spacing (3 options: only pick one)
% \doublespacing
% \singlespacing
\onehalfspacing
\boldmath
\usepackage{amsmath} 

%% ===============================================

\setlength{\droptitle}{-5em} %% Don't touch

% %%%%%%%%%%%%%%%%%%%%%%%%%%%%%%%%%%%%%%%%%%%%%%%%%%%%%%%%%%
% SET THE TITLE
% %%%%%%%%%%%%%%%%%%%%%%%%%%%%%%%%%%%%%%%%%%%%%%%%%%%%%%%%%%

% TITLE:
\title{Dissociable Systems for Social Decision Making under Risk and Time Complexity
\thanks{Report written in fulfillment of requirements for PSC 290- Decision Neuroscience, WQ-2020, UC Davis}
}

% AUTHORS:
\author{Soukhin Das\\% Name author
    \href{mailto:skndas@ucdavis.edu}{\texttt{skndas@ucdavis.edu}} %% Email author 1 
    }
    
% DATE:
\date{\today}

% %%%%%%%%%%%%%%%%%%%%%%%%%%%%%%%%%%%%%%%%%%%%%%%%%%%%%%%%%%
% %%%%%%%%%%%%%%%%%%%%%%%%%%%%%%%%%%%%%%%%%%%%%%%%%%%%%%%%%%
\begin{document}
% %%%%%%%%%%%%%%%%%%%%%%%%%%%%%%%%%%%%%%%%%%%%%%%%%%%%%%%%%%
% %%%%%%%%%%%%%%%%%%%%%%%%%%%%%%%%%%%%%%%%%%%%%%%%%%%%%%%%%%
% ABSTRACT
% %%%%%%%%%%%%%%%%%%%%%%%%%%%%%%%%%%%%%%%%%%%%%%%%%%%%%%%%%%
% %%%%%%%%%%%%%%%%%%%%%%%%%%%%%%%%%%%%%%%%%%%%%%%%%%%%%%%%%%


% %%%%%%%%%%%%%%%%%%%%%%%%%%%%%%%%%%%%%%%%%%%%%%%%%%%%%%%%%%
% %%%%%%%%%%%%%%%%%%%%%%%%%%%%%%%%%%%%%%%%%%%%%%%%%%%%%%%%%%
% BODY OF THE DOCUMENT
% %%%%%%%%%%%%%%%%%%%%%%%%%%%%%%%%%%%%%%%%%%%%%%%%%%%%%%%%%%
% %%%%%%%%%%%%%%%%%%%%%%%%%%%%%%%%%%%%%%%%%%%%%%%%%%%%%%%%%%

% --------------------
\section{Introduction}
% --------------------

Event related fMRI experimental designs have become increasingly popular in the recent years[cit] to measure the BOLD signal for visualizing human brain activity. The BOLD signal being slow and sluggish present significant issues in design and analysis of experiments. The challenge is to maximize the statistical efficiency per se to detect and estimate a signal. It is achieved by carefully choosing critical factors of the experimental design subject to constraints. A substantial improvement in efficiency is achieved by randomizing the order and timing of events in a design. This study focuses on a particular type of design where randomization cannot be employed, namely the alternating event related design which is typically used in an attention cueing experiment where events occur in a fixed and defined sequence like a trial by trial cue- target paradigm. [cit] We used a python package- fmrisim [cit] to investigate this optimization challenge and how critical factors like Stimulus Onset Asynchrony (SOA), frequency and semi-null trials affect the efficiency of an alternating design. The seamless simulation power of fmrisim to generate standardized, realistic simulation of fMRI data is used to understand how the varying parameters can be optimized to minimize the estimation of overlapping of events in alternating event related (AEr) design sequences.

The goal in AEr fMRI experiments is often one of the two: detection or estimation of the signal. Detection represents the ability to detect the difference in brain activation between different conditions or groups, for instance, detecting an activation for cue or target in contrast to the baseline or another condition. Estimation on the other hand is how accurate the shape of the evoked response is (the haemodynamic response function, HRF). We show that these two goals are opposite to each other and an increase in one of them inevitably leads to a decrease in the other one.
 


% --------------------
\section{Methodology}
% --------------------
We assumed the general linear model as in Equation 1 as the underlying model for the neural mechanisms as the signal measured with fMRI is related to the neural signal via a convolution with a haemodynamic response function(HRF).

\begin{equation}
Y=X\beta+ \epsilon,\epsilon ~ N(0,\sigma)
\end{equation}

where $Y$ is the \textit{N x 1} voxel wise BOLD time series, $X$ is the \textit{N x k} design matrix (for $k$ events) which represents the expected response, $\beta$ is the response amplitude for each condition in $X$ and $\epsilon$ is the normally distributed error term $~ N(0,\sigma$).\par

$Y$, the time course of voxel wise BOLD response is simulated using fmrisim. $Y$ is the combination of the evoked signal activity along with noise. Fmrisim can extract noise parameters directly from an fMRI dataset. The noise generated by fmrisim comprises of multiple components: drift and system noise related to the machine, auto regressive/moving average(ARMA), physiological and task noise related to the brain. In our study, we used the publicly available dataset [cit] to estimate the noise parameters.

The efficiency of estimation is inversely related to the variance of the parameter estimates. Assuming independent errors, the unbiased estimate of the parameters is given by the least squares estimation, \(\hat{\beta} = (X' X)^{-1} X' Y \). Since fMRI noise shows evidence of significant temporal autocorrelation, the errors in Equation 1 are dependent and correlated. So we used a prewhitening method so that the parameter estimate changes to,

\begin{equation}
    \hat{\beta} = ((KX)' (KX)^{-1}) (KX)' KY,
\end{equation}

where K is a decorrelating matrix such that \(KVK'\) is the identity matrix and $V$ is the correlation matrix of errors. We have the variance - covariance matrix of the parameter estimate as,

\begin{equation}
    cov(C\hat{\beta}) = \sigma^2 (Z'Z)^{-1} Z'KVK'Z(Z'Z)^{-1}
\end{equation}

where \(Z = KX\), the whitened designed matrix [cit]. Let $C$ be the contrast matrix of interest, the parameter estimate for the contrast changes to \(C\hat{\beta}\).\par

Now after prewhitening, assuming unit variance (\(\sigma^2 = 1\)), and from the property of Moore–Penrose inverse (\(A^-=(A'A)^{-1}A'\)), equation 3 reduces to

\begin{equation}
    cov(\hat{\beta}) = CZ^-KVK'(Z^-)'C'
\end{equation}

The lesser the variance of the parameter estimates, the more optimized the experimental design is. Hence, equation 5 defines the efficiency of a design.

\begin{equation}
    \xi = \frac{1} {trace\{cov(C\hat{\beta})\}}
\end{equation}
\\
\(\xi\) is the detection power (\(\xi_d\)) if X is a convolved design matrix.\\
\(\xi\) is the estimation efficiency (\(\xi_e\)) if X is a finite impulse response(FIR) matrix of the HRF.\\ In order to ensure compatibility and comparibility across the measures of \(\xi_d\) and \(\xi_e\), they are standardized as follows,

\begin{equation}
    \xi_i^* = \frac{\xi_i} {max(\xi_i)}, i = d , e
\end{equation}


% --------------------
\section{Simulations}
In alternate event related fMRI design such as an attention cueing experimental design, the event of maximum interest is the cue and the post- cue anticipatory period [cit]. So our simulations are based on detecting the cue and estimating the shape of evoked response to the cue. For simplicity, the simulations have only two events - a cue (C) followed by a target (T). Table 1 shows the event transition matrix for the designs simulated. A contrast matrix $C$ = [ 1 , 0 ] was used to assess the efficiency of detecting the cue with respect to the average baseline (\(\xi_d\)) and estimating the HRF (\(\xi_e\)). The canonical haemodynamic response function was used in all the simulations.



\begin{table}[t]
\centering
\caption{Event Transition Matrix}
\begin{tabular}{llll} 
\hline
Event & C   & T   & Example                          \\ 
\hline
C     & 0.5 & 0.5 & \multirow{2}{*}{CTCTCTCTCT....}  \\
T     & 0.5 & 0.5 &                                  \\
\hline
\end{tabular}
\end{table}

\subsection{Simulation I}
This simulation used the event transition matrix shown in Table 1 to generate the event trains. Each trial consists of a cue followed by a target. Here we demonstrate an exhaustive search over Stimulus Onset Asynchrony (SOA) and jitter and its effect on $\xi$. The efficiency of 20 x 20 = 400 design sequences was estimated as a function of lower and upper bounds of SOA. The SOA was jittered uniformly between the lower and upper bound of the SOA for that particular design sequence. Both the bounds ranged from 1 to 20s with an increment of 1. The entire simulation was iterated 100 times and the their mean efficiency was used as the population reference for each combination of bounds of SOA. The TR was 2s with 294 acquisitions, which makes every time series of an experimental design to be of 588s. The duration of HDRs simulated for each design sequence was 30s long. These acquisition and modelling parameters were used in Simulation II as well.

\subsection{Simulation II}
Previous studies have suggested that inclusion of null events in the design sequence facilitate the efficiency of a design sequence[cit]. This simulation investigates this postulation. $\xi$ was estimated as a function of proportion of null events in a design sequence. Since we are primarily focusing on optimizing the efficiency of cue events, we set targets as null i.e. in some trials the cue will not be followed by a target, we name these 'semi null trials'. The proportion is varied from 0\%  semi - null trials to 50\% semi - null trials. For this simulation, the lower bound of SOA was fixed at 1s while varying the upper bound from 1 to 20 s with an increment of 2.

\section{Results and Discussion}

The heat-maps in Fig. 1(a) and Fig. 1(b) generated from Simulation I show $\xi_d$ and $\xi_e$ respectively as a function of the lower and upper bounds of jittered SOA. The results replicate the findings of [cit]. The results of Fig. 1(a) reflect that an SOA jittered uniformly between 5-9s produces maximum $\xi_e$ for cues. This gives a mean SOA of 7s, thus giving an average time of 14s between two cues.\par
On the other hand, Fig. 1(b) suggest that the maximum $\xi_e$ can be obtained when the bounds of jitter for SOA is maximum. It is evident that the design for statistically efficient detection is not the best one for efficiently estimation of the signal. The ability of fmrisim to generate non- linear patterns of saturation in the signal helped in revealing how $\xi_d$ falls off dramatically at shorter SOAs when it becomes difficult to contrast the signal evoked due to cues with respect to the saturated baseline. These results are based on detecting differences in brain response to one event with respect to the baseline. If the goal is to detect other contrasts, a different efficiency heat-map would be expected.\\

The results obtained from Simulation II are presented in Fig. 2(a) and Fig. 2(b). Based on these results, the inclusion of null targets increase detection power ($\xi_d$) of cues at shorter SOAs and is directly related to the proportion of inclusion. It should be noted that as the SOA is increased, different proportions of semi null trials result in very similar efficiency measures converging into one another. In other words, semi null trials do not aid the detection at longer SOAs.\par
% --------------------
\section{Discussion and Conclusion}
% --------------------


\medskip

\bibliography{references.bib} 

\end{document}